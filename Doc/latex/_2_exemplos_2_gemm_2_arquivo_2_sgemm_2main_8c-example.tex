\section{/\+Exemplos/\+Gemm/\+Arquivo/\+Sgemm/main.\+c}
Exemplo de multiplicação de matrizes utilizando a lib\+O\+P\+CL


\begin{DoxyCodeInclude}
\textcolor{preprocessor}{#pragma OPENCL EXTENSION cl\_khr\_fp64 : enable // Usar variáveis do tipo Double}

\textcolor{preprocessor}{#include <stdlib.h>}
\textcolor{preprocessor}{#include <stdio.h>}
\textcolor{preprocessor}{#include <string.h>}
\textcolor{preprocessor}{#include <ctype.h>}
\textcolor{preprocessor}{#include <math.h>}
\textcolor{preprocessor}{#include <ctype.h>}
\textcolor{preprocessor}{#include <libopcl.h>}
\textcolor{preprocessor}{#include <omp.h>}


\textcolor{comment}{// includes do OpenCL}
\textcolor{preprocessor}{#include <CL/cl.h>}

\textcolor{preprocessor}{#define MAX\_SOURCE\_SIZE (0x100000)}
\textcolor{comment}{//Tamanho do work-Group}
\textcolor{preprocessor}{#define BSIZE 32}

\textcolor{comment}{//Função para multiplicação de matrizes na CPU}
\textcolor{keywordtype}{void} gemm\_CPU(\textcolor{keywordtype}{float} *a, \textcolor{keywordtype}{float}* b, \textcolor{keywordtype}{float} *c, \textcolor{keywordtype}{int} size, \textcolor{keywordtype}{float} alfa, \textcolor{keywordtype}{float} beta);
\textcolor{comment}{//Função para multiplicação de matrizes usando OpenCL}
\textcolor{keywordtype}{void} gemm\_OpenCL(\textcolor{keywordtype}{float} *a, \textcolor{keywordtype}{float}* b, \textcolor{keywordtype}{float} *c, \textcolor{keywordtype}{int} size, \textcolor{keywordtype}{int} t, \textcolor{keywordtype}{float} alfa, \textcolor{keywordtype}{float} beta);
\textcolor{comment}{//Função para comparação das matrizes geradas}
\textcolor{keywordtype}{float} comparaMatrizes(\textcolor{keywordtype}{float}* a, \textcolor{keywordtype}{float}* b, \textcolor{keywordtype}{int} size);
\textcolor{keywordtype}{void} printMatrix(\textcolor{keywordtype}{float} *a, \textcolor{keywordtype}{int} size);

\textcolor{keywordtype}{int} main(\textcolor{keywordtype}{int} argc, \textcolor{keywordtype}{char} *argv[])
\{

    \textcolor{comment}{// Dados do Host}
    \textcolor{keywordtype}{float} *A = NULL;  \textcolor{comment}{// Array de entrada}
    \textcolor{keywordtype}{float} *B = NULL;  \textcolor{comment}{// Array de entrada}
    \textcolor{keywordtype}{float} *C\_host = NULL;  \textcolor{comment}{// Array de saída}
    \textcolor{keywordtype}{float} *C\_device = NULL;  \textcolor{comment}{// Array de saída}

    \textcolor{keywordtype}{float} start, stop, t\_s, t\_p;
    \textcolor{keywordtype}{float} error;

    \textcolor{keywordtype}{float} alfa = 0.3, beta = 0.4;
    \textcolor{comment}{// Tamanho da Matriz(square matrix)}
    \textcolor{keywordtype}{int} mSize = 1024;
    \textcolor{keywordflow}{if}(argc >= 2)\textcolor{comment}{//Receber tamanho da matriz por linha de comando, caso não receber nada o tamanho padrão é
       1024x1024}
        mSize = atoi(argv[1]);

    \textcolor{comment}{// Elementos em cada array}
    \textcolor{keyword}{const} \textcolor{keywordtype}{int} elements = mSize*mSize;

    \textcolor{comment}{// Calcula o tamanho dos dados}
    \textcolor{keywordtype}{size\_t} datasize = \textcolor{keyword}{sizeof}(float)*elements;

    \textcolor{comment}{//Aloca espaço para os dados de Entrada/Saída}
    A = (\textcolor{keywordtype}{float}*)malloc(datasize);
    B = (\textcolor{keywordtype}{float}*)malloc(datasize);
    C\_host = (\textcolor{keywordtype}{float}*)malloc(datasize);
    C\_device = (\textcolor{keywordtype}{float}*)malloc(datasize);

    \textcolor{comment}{// Incializa os dados de entrada}
    \textcolor{keywordflow}{for}(\textcolor{keywordtype}{int} i = 0; i < elements; i++) \{
        A[i] = 1.0;
        B[i] = 2.0;
        C\_host[i] = 3.0;
        C\_device[i] = 3.0;
    \}

    \textcolor{comment}{//Execução do código Serial}
    start = omp\_get\_wtime();
    gemm\_CPU(A,B,C\_host,mSize, alfa, beta);
    stop = omp\_get\_wtime();
    t\_s = stop - start;
    printf(\textcolor{stringliteral}{"Result Ok! time - %f\(\backslash\)n"}, t\_s);

    \textcolor{comment}{//Execução do código Paralelo}
    start = omp\_get\_wtime();
    gemm\_OpenCL(A, B, C\_device, mSize, 0,  alfa,  beta);

    stop = omp\_get\_wtime();
    t\_p = stop - start;

    \textcolor{comment}{//Comparação dos dados e de tempo dos dois códigos}
    error = comparaMatrizes(C\_host, C\_device, mSize);
    printf(\textcolor{stringliteral}{"Error - %f\(\backslash\)n"}, error);
    printf(\textcolor{stringliteral}{"Result Ok! time - %f\(\backslash\)n"}, t\_p);
    printf(\textcolor{stringliteral}{"Result Ok! speedup - %f\(\backslash\)n"}, t\_s/t\_p);

    \textcolor{comment}{// Libera os recursos do Host}
    free(A);
    free(B);
    free(C\_host);
    free(C\_device);
\}


\textcolor{keywordtype}{void} gemm\_CPU(\textcolor{keywordtype}{float} *a, \textcolor{keywordtype}{float}* b, \textcolor{keywordtype}{float} *c, \textcolor{keywordtype}{int} size, \textcolor{keywordtype}{float} alfa, \textcolor{keywordtype}{float} beta)
\{
    \textcolor{keywordtype}{int} i, j, k;
    \textcolor{keywordtype}{float} cValue;

    \textcolor{keywordflow}{for}(i=0; i<size; i++)
        \textcolor{keywordflow}{for}(j=0; j<size; j++)\{
            cValue = c[i*size+j];
            \textcolor{keywordflow}{for}(k=0; k<size; k++)
                cValue += a[i*size+k]*b[k*size+j];
            c[i*size+j] = alfa* cValue + beta*c[i*size+j];
        \}

        \textcolor{keywordflow}{return};
\}
\textcolor{keywordtype}{void} gemm\_OpenCL(\textcolor{keywordtype}{float} *a, \textcolor{keywordtype}{float}* b, \textcolor{keywordtype}{float} *c, \textcolor{keywordtype}{int} size, \textcolor{keywordtype}{int} t, \textcolor{keywordtype}{float} alfa, \textcolor{keywordtype}{float} beta)
\{
    \textcolor{comment}{// Elementos em cada array}
    \textcolor{keyword}{const} \textcolor{keywordtype}{int} elements = size*size;
    \textcolor{keywordtype}{int} error = 0;
    \textcolor{comment}{// Calcula o tamanho dos elementos}
    \textcolor{keywordtype}{size\_t} datasize = \textcolor{keyword}{sizeof}(float)*elements;

    \textcolor{comment}{// Usado para checar a saída do OpenCL}
    cl\_int status;


    \textcolor{comment}{//Leitura do arquivo onde está o kernel}
    FILE *fp;
    \textcolor{keywordtype}{char} *source\_str;
    \textcolor{keywordtype}{size\_t} source\_size;

    \textcolor{comment}{//Carregando o arquivo com o Kernel}
    fp = fopen(\textcolor{stringliteral}{"dgemm-kernelPrivate.cl"}, \textcolor{stringliteral}{"r"});
    \textcolor{keywordflow}{if} (!fp) \{
        fprintf(stderr, \textcolor{stringliteral}{"Failed to load kernel.\(\backslash\)n"});
        exit(1);
    \}

    fseek(fp, 0, SEEK\_END);\textcolor{comment}{//Envia o cabeçote de leitura para o fim do arquivo.}
    \textcolor{keywordtype}{long} fSize = ftell(fp);\textcolor{comment}{//Retorna o tamanho do arquivo em bytes.}
    rewind(fp);\textcolor{comment}{//Retorna a leitura para o inicio do arquivo.}

    source\_str = (\textcolor{keywordtype}{char}*)malloc((fSize+1)*\textcolor{keyword}{sizeof}(char));\textcolor{comment}{// Cria uma string com o tamanho do arquivo}
    source\_size = fread( source\_str, 1, (fSize+1)*\textcolor{keyword}{sizeof}(\textcolor{keywordtype}{char}), fp);\textcolor{comment}{//Copia o conteúdo do kernel para a
       variável pré alocada}
    source\_str[fSize] = \textcolor{charliteral}{'\(\backslash\)0'};\textcolor{comment}{//Marca o fim do arquivo.}
    fclose( fp );\textcolor{comment}{//Fecha o arquivo.}

    \textcolor{comment}{//printf("%F\(\backslash\)n", c[0]);}

    \textcolor{comment}{//
      ----------------------------------------------------------------------------------------------------------}
    \textcolor{comment}{// STEP 1: Descobrir e inicializar as plataformas e Devices, e criar contexto e Fila de Comando}
    \textcolor{comment}{//
      ----------------------------------------------------------------------------------------------------------}

    lopcl_Init(lopcl_POCL, lopcl_DEVICE_CPU);

    \textcolor{comment}{//
      ----------------------------------------------------------------------------------------------------------}
    \textcolor{comment}{// STEP 2: Criar buffers do device e Escrever os dados do host para os lopcl\_DEVICES de buffers}
    \textcolor{comment}{//-----------------------------------------------------
       -----------------------------------------------------}


    cl\_mem bufferA;  \textcolor{comment}{// Array de entrada do device}
    cl\_mem bufferB;  \textcolor{comment}{// Array de entrada do device}
    cl\_mem bufferC;  \textcolor{comment}{// Array de saída do device}

    \textcolor{comment}{// Use lopcl\_CreateBuffer() para criar o objeto (d\_A)}
    \textcolor{comment}{// Isso contêm os dados do Array A do Host.}
    bufferA = lopcl_CreateBuffer(datasize, CL\_MEM\_READ\_ONLY, CL\_FALSE, a);

    \textcolor{comment}{// Use lopcl\_CreateBuffer() para criar o objeto (d\_B)}
    \textcolor{comment}{// Isso contêm os dados do Array B do Host.}
    bufferB = lopcl_CreateBuffer(datasize, CL\_MEM\_READ\_ONLY, CL\_FALSE, b);

    \textcolor{comment}{// Use lopcl\_CreateBuffer() para criar o objeto (d\_B)}
    \textcolor{comment}{// com espaço suficiente para "segurar" os dados de saída.}
    bufferC = lopcl_CreateBuffer(datasize, CL\_MEM\_READ\_WRITE, CL\_FALSE, c);

    \textcolor{comment}{//---------------------------------------------------------------}
    \textcolor{comment}{// STEP 3: Cria e compila o programa, e cria o kernel}
    \textcolor{comment}{//----------------------------------------------------- ---------}

    \textcolor{keywordflow}{if}(t==0)
        lopcl_CreateProgram((\textcolor{keyword}{const} \textcolor{keywordtype}{char}**)&source\_str, \textcolor{stringliteral}{"dgemm"});


    \textcolor{comment}{//-----------------------------------------------------}
    \textcolor{comment}{// STEP 4: Define os argumentos do kernel}
    \textcolor{comment}{//-----------------------------------------------------}

    \textcolor{comment}{// Associa os buffers de entrada e saida ao kernel}
    \textcolor{comment}{// usando lopcl\_SetKernelArg()}

    lopcl_SetKernelArg(0, \textcolor{keyword}{sizeof}( \textcolor{keywordtype}{int}), &size);

    lopcl_SetKernelArg(1, \textcolor{keyword}{sizeof}(\textcolor{keywordtype}{int}), &size);

    lopcl_SetKernelArg(2, \textcolor{keyword}{sizeof}(\textcolor{keywordtype}{int}), &size);

    lopcl_SetKernelArg(3, \textcolor{keyword}{sizeof}(\textcolor{keywordtype}{float}), &alfa);

    lopcl_SetKernelArg(4, \textcolor{keyword}{sizeof}(cl\_mem), &bufferA);

    lopcl_SetKernelArg(5, \textcolor{keyword}{sizeof}(\textcolor{keywordtype}{int}), &size);

    lopcl_SetKernelArg(6, \textcolor{keyword}{sizeof}(cl\_mem), &bufferB);

    lopcl_SetKernelArg(7, \textcolor{keyword}{sizeof}(\textcolor{keywordtype}{int}), &size);

    lopcl_SetKernelArg(8, \textcolor{keyword}{sizeof}(\textcolor{keywordtype}{float}), &beta);

    lopcl_SetKernelArg(9, \textcolor{keyword}{sizeof}(cl\_mem), &bufferC);

    lopcl_SetKernelArg(10, \textcolor{keyword}{sizeof}(\textcolor{keywordtype}{int}), &size);


    \textcolor{comment}{//--------------------------------------------------------------}
    \textcolor{comment}{// STEP 5: Configurar a estrutura dos work itens e work groups}
    \textcolor{comment}{//----------------------------------------------------- --------}

    \textcolor{keywordtype}{size\_t} globalWorkSize[2];
    \textcolor{comment}{// Esses são os elementos de work-items}
    globalWorkSize[0] = size/32;
    globalWorkSize[1] = size/32;

    \textcolor{keywordtype}{size\_t} localWorkSize[2];

    localWorkSize[0] = BSIZE;
    localWorkSize[1] = BSIZE;

    \textcolor{comment}{//-----------------------------------------------------}
    \textcolor{comment}{// STEP 6: Empilha o Kernel para execução e Ler a saida do buffer de saída e salva no host}
    \textcolor{comment}{//-----------------------------------------------------}

    \textcolor{comment}{// Execute the kernel by using}
    \textcolor{comment}{// clEnqueueNDRangeKernel().}
    \textcolor{comment}{// 'globalWorkSize' is the 1D dimension of the}
    \textcolor{comment}{// work-items}
    
    lopcl_clEnqueueNDRangeKernel(2, NULL, globalWorkSize, localWorkSize, bufferC, CL\_TRUE, datasize, c);

    \textcolor{comment}{//-----------------------------------------------------}
    \textcolor{comment}{// STEP 8: libera os recursos}
    \textcolor{comment}{//-----------------------------------------------------}

    \textcolor{comment}{//Free OpenCL resources}
    clReleaseKernel(lopcl_KERNEL );
    clReleaseProgram(lopcl_PROGRAM);
    clReleaseCommandQueue(lopcl_CMDQUEUE);
    clReleaseContext(lopcl_CONTEXT);
    clReleaseMemObject(bufferA);
    clReleaseMemObject(bufferB);
    clReleaseMemObject(bufferC);
    lopcl_Finalize();
    free(source\_str);


    \textcolor{keywordflow}{return};

\}

\textcolor{keywordtype}{float} comparaMatrizes(\textcolor{keywordtype}{float}* a, \textcolor{keywordtype}{float}* b, \textcolor{keywordtype}{int} size)
\{
    \textcolor{keywordtype}{float} max = 0;
    \textcolor{keywordtype}{int} i, tSize = size*size;
    \textcolor{keywordflow}{for}(i=0; i<tSize; i++)
        \textcolor{keywordflow}{if}(max < fabs(a[i] - b[i]))
            max =  fabs(a[i] - b[i]);

        \textcolor{keywordflow}{return} max;

\}

\textcolor{keywordtype}{void} printMatrix(\textcolor{keywordtype}{float} *a, \textcolor{keywordtype}{int} size)
\{
    \textcolor{keywordtype}{int} i,j;
    \textcolor{keywordflow}{for}(i=0;i<size;i++)\{
        \textcolor{keywordflow}{for}(j=0; j<size; j++)
            printf(\textcolor{stringliteral}{"%f "}, a[i*size +j]);
        printf(\textcolor{stringliteral}{"\(\backslash\)n"});
    \}
\}
\end{DoxyCodeInclude}
 